\documentclass[12pt]{article}
\usepackage{pmmeta}
\pmcanonicalname{SpinGroups}
\pmcreated{2013-03-22 18:22:44}
\pmmodified{2013-03-22 18:22:44}
\pmowner{bci1}{20947}
\pmmodifier{bci1}{20947}
\pmtitle{spin groups}
\pmrecord{39}{41021}
\pmprivacy{1}
\pmauthor{bci1}{20947}
\pmtype{Definition}
\pmcomment{trigger rebuild}
\pmclassification{msc}{82D30}
\pmclassification{msc}{83C60}
\pmclassification{msc}{81R25}
\pmclassification{msc}{15A66}
\pmclassification{msc}{53C27}
\pmsynonym{spinor in quantum physics}{SpinGroups}
%\pmkeywords{spin groups}
%\pmkeywords{spin symmetry}
%\pmkeywords{spinor in quantum physics}
%\pmkeywords{SUSY}
%\pmkeywords{Standard Model in current physics}
%\pmkeywords{Clifford algebra}
%\pmkeywords{Spinor and twistor methods; Newman-Penrose formalism}
%\pmkeywords{Pauli spin matrices}
%\pmkeywords{Dirac matrix}
%\pmkeywords{Spin and Spin$^c$ geometry}
%\pmkeywords{spin glasses}
%\pmkeywords{spin waves}
\pmrelated{CliffordAlgebra2}
\pmrelated{ExactSequence}
\pmrelated{CategoricalSequence}
\pmrelated{ExamplesOfGroups}
\pmrelated{NoncommutativeGeometry}
\pmrelated{ElieJosephCartan}
\pmdefines{spin group}
\pmdefines{spin symmetry}
\pmdefines{$Spin(n)$}
\pmdefines{$SO(n)$}
\pmdefines{$Spin(3)$}
\pmdefines{$Spin(4)$}
\pmdefines{$Sp(1)$}
\pmdefines{short exact sequence of Lie groups}

% this is the default PlanetMath preamble.  as your knowledge
% of TeX increases, you will probably want to edit this, but
% it should be fine as is for beginners.

% almost certainly you want these
\usepackage{amssymb}
\usepackage{amsmath}
\usepackage{amsfonts}

% used for TeXing text within eps files
%\usepackage{psfrag}
% need this for including graphics (\includegraphics)
%\usepackage{graphicx}
% for neatly defining theorems and propositions
%\usepackage{amsthm}
% making logically defined graphics
%%%\usepackage{xypic}

% there are many more packages, add them here as you need them

% define commands here
\usepackage{amsmath, amssymb, amsfonts, amsthm, amscd, latexsym}
%%\usepackage{xypic}
\usepackage[mathscr]{eucal}

\setlength{\textwidth}{6.5in}
%\setlength{\textwidth}{16cm}
\setlength{\textheight}{9.0in}
%\setlength{\textheight}{24cm}

\hoffset=-.75in     %%ps format
%\hoffset=-1.0in     %%hp format
\voffset=-.4in

\theoremstyle{plain}
\newtheorem{lemma}{Lemma}[section]
\newtheorem{proposition}{Proposition}[section]
\newtheorem{theorem}{Theorem}[section]
\newtheorem{corollary}{Corollary}[section]

\theoremstyle{definition}
\newtheorem{definition}{Definition}[section]
\newtheorem{example}{Example}[section]
%\theoremstyle{remark}
\newtheorem{remark}{Remark}[section]
\newtheorem*{notation}{Notation}
\newtheorem*{claim}{Claim}

\renewcommand{\thefootnote}{\ensuremath{\fnsymbol{footnote%%@
}}}
\numberwithin{equation}{section}

\newcommand{\Ad}{{\rm Ad}}
\newcommand{\Aut}{{\rm Aut}}
\newcommand{\Cl}{{\rm Cl}}
\newcommand{\Co}{{\rm Co}}
\newcommand{\DES}{{\rm DES}}
\newcommand{\Diff}{{\rm Diff}}
\newcommand{\Dom}{{\rm Dom}}
\newcommand{\Hol}{{\rm Hol}}
\newcommand{\Mon}{{\rm Mon}}
\newcommand{\Hom}{{\rm Hom}}
\newcommand{\Ker}{{\rm Ker}}
\newcommand{\Ind}{{\rm Ind}}
\newcommand{\IM}{{\rm Im}}
\newcommand{\Is}{{\rm Is}}
\newcommand{\ID}{{\rm id}}
\newcommand{\GL}{{\rm GL}}
\newcommand{\Iso}{{\rm Iso}}
\newcommand{\Sem}{{\rm Sem}}
\newcommand{\St}{{\rm St}}
\newcommand{\Sym}{{\rm Sym}}
\newcommand{\SU}{{\rm SU}}
\newcommand{\Tor}{{\rm Tor}}
\newcommand{\U}{{\rm U}}

\newcommand{\A}{\mathcal A}
\newcommand{\Ce}{\mathcal C}
\newcommand{\D}{\mathcal D}
\newcommand{\E}{\mathcal E}
\newcommand{\F}{\mathcal F}
\newcommand{\G}{\mathcal G}
\newcommand{\Q}{\mathcal Q}
\newcommand{\R}{\mathcal R}
\newcommand{\cS}{\mathcal S}
\newcommand{\cU}{\mathcal U}
\newcommand{\W}{\mathcal W}

\newcommand{\bA}{\mathbb{A}}
\newcommand{\bB}{\mathbb{B}}
\newcommand{\bC}{\mathbb{C}}
\newcommand{\bD}{\mathbb{D}}
\newcommand{\bE}{\mathbb{E}}
\newcommand{\bF}{\mathbb{F}}
\newcommand{\bG}{\mathbb{G}}
\newcommand{\bK}{\mathbb{K}}
\newcommand{\bM}{\mathbb{M}}
\newcommand{\bN}{\mathbb{N}}
\newcommand{\bO}{\mathbb{O}}
\newcommand{\bP}{\mathbb{P}}
\newcommand{\bR}{\mathbb{R}}
\newcommand{\bV}{\mathbb{V}}
\newcommand{\bZ}{\mathbb{Z}}

\newcommand{\bfE}{\mathbf{E}}
\newcommand{\bfX}{\mathbf{X}}
\newcommand{\bfY}{\mathbf{Y}}
\newcommand{\bfZ}{\mathbf{Z}}

\renewcommand{\O}{\Omega}
\renewcommand{\o}{\omega}
\newcommand{\vp}{\varphi}
\newcommand{\vep}{\varepsilon}

\newcommand{\diag}{{\rm diag}}
\newcommand{\grp}{{\mathbb G}}
\newcommand{\dgrp}{{\mathbb D}}
\newcommand{\desp}{{\mathbb D^{\rm{es}}}}
\newcommand{\Geod}{{\rm Geod}}
\newcommand{\geod}{{\rm geod}}
\newcommand{\hgr}{{\mathbb H}}
\newcommand{\mgr}{{\mathbb M}}
\newcommand{\ob}{{\rm Ob}}
\newcommand{\obg}{{\rm Ob(\mathbb G)}}
\newcommand{\obgp}{{\rm Ob(\mathbb G')}}
\newcommand{\obh}{{\rm Ob(\mathbb H)}}
\newcommand{\Osmooth}{{\Omega^{\infty}(X,*)}}
\newcommand{\ghomotop}{{\rho_2^{\square}}}
\newcommand{\gcalp}{{\mathbb G(\mathcal P)}}

\newcommand{\rf}{{R_{\mathcal F}}}
\newcommand{\glob}{{\rm glob}}
\newcommand{\loc}{{\rm loc}}
\newcommand{\TOP}{{\rm TOP}}

\newcommand{\wti}{\widetilde}
\newcommand{\what}{\widehat}

\renewcommand{\a}{\alpha}
\newcommand{\be}{\beta}
\newcommand{\ga}{\gamma}
\newcommand{\Ga}{\Gamma}
\newcommand{\de}{\delta}
\newcommand{\del}{\partial}
\newcommand{\ka}{\kappa}
\newcommand{\si}{\sigma}
\newcommand{\ta}{\tau}
\newcommand{\med}{\medbreak}
\newcommand{\medn}{\medbreak \noindent}
\newcommand{\bign}{\bigbreak \noindent}
\newcommand{\lra}{{\longrightarrow}}
\newcommand{\ra}{{\rightarrow}}
\newcommand{\rat}{{\rightarrowtail}}
\newcommand{\oset}[1]{\overset {#1}{\ra}}
\newcommand{\osetl}[1]{\overset {#1}{\lra}}
\newcommand{\hr}{{\hookrightarrow}}

\begin{document}
\section{Spin groups}
\subsection{Description}

Spins and spin group mathematics are important subjects both in theoretical physics and mathematics.
In physics, the term {\em spin `groups'} is often used with the broad meaning of a  
collection of coupled, or interacting spins, and thus covers the broad `spectrum' of spin clusters
ranging from gravitons (as in spin networks and spin foams, for example) to `up' ($u$) and `down' ($d$) quark spins
(fermions) coupled by gluons in nuclei (as treated in quantum chromodynamics or theoretical nuclear physics),
and electron spin Cooper pairs (regarded as bosons) in low-temperature superconductivity. 
On the other hand, in relation to \emph{quantum symmetry}, \emph{spin groups} are defined in 
quantum mechanics and quantum field theories (QFT) in a precise, mathematical (algebraic) sense as properly defined  
groups, as introduced next. (In a semi-classical approach, the related concept of a \emph{spinor} has been introduced and studied in depth by \'E. Cartan, who found that with his definition of spinors the (special) relativistic Lorentz covariance properties were not recovered, or applicable.)
\bigbreak
\begin{definition}
In the mathematical, precise sense of the term, a \emph{spin group}
--as for example the Lie group $Spin(n)$-- is defined as a \emph{double cover of the special orthogonal (Lie) group $SO(n)$} satisfying the additional condition that there exists the \emph{short exact sequence of Lie groups}:

$$ 1 \to \mathbb{Z}_2 \to Spin(n) \to SO(n) \to 1 $$

Alternatively one can say that the above exact sequence of Lie groups defines the spin group $Spin(n)$. 
Furthermore, \emph{$Spin(n)$} can also be defined as the \emph{proper subgroup (or groupoid) of the invertible elements
in the \PMlinkname{Clifford algebra}{CliffordAlgebra2}} $\mathbb{C}l(n)$; (when defined as a double cover this should be $Cl_{p, q}(R)$, a \PMlinkname{Clifford algebra}{CliffordAlgebra2} built up from an orthonormal basis of $n = p + q$ mutually orthogonal vectors under addition and multiplication, $p$ of which have norm +1 and $q$ of which have norm $-1$, as further explained in the \PMlinkname{spinor definition}{Spinor}). 

Note also that other spin groups such as \emph{$Spin ~ d$} (ref. \cite{SW1999}) are mathematically defined, and also important, in \PMlinkname{QFT}{QFTOrQuantumFieldTheories}.  
\end{definition}

{\bf Important examples of $Spin(n)$ and quantum symmetries: there exist the following isomorphisms:}

\begin{enumerate}
\item $Spin(1) \cong O(1) $  

\item $Spin(2) \cong U(1)\cong SO(2) $

\item $Spin(3) \cong Sp(1)\cong SU(2)$ 

\item $Spin(4) \cong  Sp(1) \times Sp(1)$

\item $Spin(5) \cong Sp(2) $

\item $Spin(6) \cong SU(4) $

\end{enumerate}

Thus, the symmetry groups in the Standard Model (SUSY) of current Physics can also be written as : 
$Spin(2) \times Spin(3) \times SU(3)$. 

{\bf Remarks}
\begin{itemize}
\item In modern Physics, non-Abelian spin groups are also defined, as for example, spin quantum groups
and spin quantum groupoids.

\item An extension of the concepts of spin group and spinor, is the notion of a `twistor', 
a mathematical concept introduced by Sir Roger Penrose, generally with distinct symmetry/mathematical properties
from those of spin groups, such as those defined above.

\subsection{The Fundamental Groups of $Spin(p,q)$}

With the usual notation, the fundamental groups $\pi_1(Spin(p,q))$ are as follows:

\begin{enumerate}

\item $\left\{0\right\}$ , for $(p,q)= (1,1)$ and $(p,q)= (1,0)$;

\item $\left\{0\right\}$ , if $p > 2$ and $q = 0,1$;

\item $\mathbb{Z}$  for $(p,q)= (2,0)$ and $(p,q)= (2,1)$;

\item $\mathbb{Z} \times \mathbb{Z}$ for $(p,q)= (2,2)$;

\item $\mathbb{Z}$ for $ p>2, q =2 $ 

\item $\mathbb{Z}_2$ for $ p>2, q >2 $ 

\end{enumerate}




\end{itemize}

\begin{thebibliography}{9}

\bibitem{AABB1970}
A. Abragam and B. Bleaney. {\em Electron Paramagnetic Resonance of Transition Ions}. 1970.
Clarendon Press: Oxford, ({\em dedicated to J. H. Van Vleck}), pp. 911. 

\bibitem{PWA-HS1955}
P.W. Anderson and H. Suhl. 1955. {\em Phys. Rev.},  {\bf 100}:1788-1795.

\bibitem{JFD1956}
J.F. Dyson., 1956. General Theory of Spin Wave interactions., {\em Phys. Rev.}, {\bf 102}:1217-1228. 

\bibitem{SW1999}
S. Weinberg. 1999. \emph{Quantum Theory of Fields}, vol. 1, Cambridge University Press: Cambridge, UK.

\bibitem{ICB1980}
I.C. Baianu et al. 1980. Ferromagnetic Resonance and Spin Wave Excitations in Metallic Glasses., 
{\em J. Phys. Chem. Solids.}, {\bf 40}: 941-950.

\bibitem{ICB1981}
I.C. Baianu et al. 1981. Nuclear Magnetic Resonance Spin-Echo Responses of Dipolar Coupled Spin -1/2 Triads 
(Groups in Solids.), {\em J. Magn. Resonance.}, {\bf 43}: 101-111.

\end{thebibliography}

%%%%%
%%%%%
\end{document}
